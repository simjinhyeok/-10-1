\chapter{結論}
\label{chap:conclusion}
\minitoc

\thispagestyle{empty}

\newpage
%%%%%%%%%%%%%%%%%%%%%%%%%%%%%%%%%%%%%%%%%%%%%%%%%%%%%%%%%%%%%%%%%%%%%%%%%%%%%%%%%%%%%%%%%%%%%%%%%%%%%%%%%%%%%%%%%%%%%%%%%%%%%%%%%%%
\section{結論}
\label{sec:conclusion_chap6}

本論文では,理学療法士の技能を活用し,片麻痺患者のリハビリを日常的に支援する機器を提案した.\\

第\ref{chap:introduction}章にて,片麻痺患者の起立動作のリハビリの重要性を示し,これまでの関連研究では,ヒトの運動のメカニズムを理解した上で動作を支援していないことを述べた.これに基づき本研究の最終目的を,ヒトの運動のメカニズムを理解し,片麻痺患者のリハビリを日常的に支援する機器を開発と設定した.その上で,本研究の目的は理学療法士の介入による片麻痺患者への影響を調査,理学療法士の技能を解析,理学療法士の技能を活用した,起立動作の支援機器の開発の三つであるとした.\\

第\ref{chap:approach}章にて,目的を実現するためにと,それらを実現するためのアプローチについて述べた.
\\

第\ref{chap:investigation}章にて,理学療法士のリハビリテーションによる片麻痺患者への影響について詳しく説明した.\\

第\ref{chap:analysys}章にて,理学療法士のリハビリテーションの技能について詳しく説明した.\\

第\ref{chap:development}章にて,起立動作の支援機器への応用と有効性を検証するために行った実験について述べた.\\

本論文により,片麻痺患者のリハビリに有効な起立動作の支援機器が確立された.

\textcolor{red}{改ページする予定です.}
%\clearpage

%%%%%%%%%%%%%%%%%%%%%%%%%%%%%%%%%%%%%%%%%%%%%%%%%%%%%%%%%%%%%%%%%%%%%%%%%%%%%%%%%%%%%%%%%%%%%%%%%%%%%%%%%%%%%%%%%%%%%%%%%%%%%%%%%%%
\section{今後の展望}
\label{sec:future_work}

本論文では,片麻痺患者のリハビリを日常的に支援する機器を開発した.さらなる有効性の検証のためには,片麻痺患者への支援機器の適用が求められる.\\

このような課題を解決することで,本提案手法は片麻痺患者だけでなく,足腰の弱くなった高齢者への支援機器として幅広く応用可能になることが期待できる.

\clearpage
%%%%%%%%%%%%%%%%%%%%%%%%%%%%%%%%%%%%%%%%%%%%%%%%%%%%%%%%%%%%%%%%%%%%%%%%%%%%%%%%%%%%%%%%%%%%%%%%%%%%%%%%%%%%%%%%%%%%%%%%%%%%%%%%%%%
%%% Local Variables:
%%% mode: katex
%%% TeX-master: "../thesis"
%%% End:
