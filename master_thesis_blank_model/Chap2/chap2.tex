\chapter{本研究のアプローチ}
\label{chap:approach}
\minitoc

\thispagestyle{empty}

\newpage
%%%%%%%%%%%%%%%%%%%%%%%%%%%%%%%%%%%%%%%%%%%%%%%%%%%%%%%%%%%%%%%%%%%%%%%%%%%%%%%%%%%%%%%%%%%%%%%%%%%%%%%%%%%%%%%%%%%%%%%%%%%%%%%%%%%
\section{はじめに}
\label{sec:intro_chap2}

本章では,ヒトの運動のメカニズムを理解し,片麻痺患者の回復期のリハビリテーション(以下リハビリ)を日常的に支援する機器を開発するアプローチについて述べる.

%\ref{sec:problem_setteing_chap2}節にて,本研究における問題設定について述べる.

\ref{sec:approach_chap2}節にて,本手法のアプローチと解決すべき課題について述べる.
\ref{subsec:investigate_chap2}項にて,理学療法士のリハビリテーションにおける片麻痺患者への影響の調査について述べる.
\ref{subsec:analysys_chap2}項にて,理学療法士の技能を解析するアプローチについて述べる.
\ref{subsec:development_chap2}項にて,理学療法士の技能を応用する支援機器の開発のアプローチについて述べる.

\ref{sec:outro_chap2}節にて,本章のまとめを述べる.

\clearpage

%%%%%%%%%%%%%%%%%%%%%%%%%%%%%%%%%%%%%%%%%%%%%%%%%%%%%%%%%%%%%%%%%%%%%%%%%%%%%%%%%%%%%%%%%%%%%%%%%%%%%%%%%%%%%%%%%%%%%%%%%%%%%%%%%%%

%\section{問題設定}
%\label{sec:problem_setteing_chap2}
%第1章にて,が重要であることを述べた.\\
%
%また,\ref{sec:related_research_chap1}節にて,を述べた.などが例として挙げられる.\\
%
%また,同様に\ref{sec:related_research_chap1}節にて,であることを述べた.従って,本研究では利用する.\\
%
%上記を踏まえて,以下に本研究の問題設定をまとめる.
%\begin{itemize}
%\item 患者への影響を調査
%\item 技能を調査
%\item 支援機器を開発
%\end{itemize}
%
%このような問題設定の下,手法を提案する.
%
%
%\clearpage


%%%%%%%%%%%%%%%%%%%%%%%%%%%%%%%%%%%%%%%%%%%%%%%%%%%%%%%%%%%%%%%%%%%%%%%%%%%%%%%%%%%%%%%%%%%%%%%%%%%%%%%%%%%%%%%%%%%%%%%%%%%%%%%%%%%
\section{アプローチ}
\label{sec:approach_chap2}
\ref{sec:objective_chap1}節にて述べた通り,本研究の目的は理学療法士の介入による片麻痺患者への影響を調査,理学療法士の技能を解析,理学療法士の技能を活用した,起立動作の支援機器の開発の三つである.
それぞれの目的を達成するためのアプローチを次節で述べていく.\\

%\clearpage
%%%%%%%%%%%%%%%%%%%%%%%%%%%%%%%%%%%%%%%%%%%%%%%%%%%%%%%%%%%%%%%%%%%%%%%%%%%%%%%%%%%%%%%%%%%%%%%%%%%%%%%%%%%%%%%%%%%%%%%%%%%%%%%%%%%
\subsection{片麻痺患者への影響の調査}
\label{subsec:investigate_chap2}

理学療法士のリハビリテーションにより片麻痺患者にどのような影響があるのかを調査することを前述した.本項では,片麻痺患者への影響の調査方法について説明する.\\

起立動作は,年齢,関節角度,床反力,関節トルク,重心軌道,表面筋電図など多くの指標により調べられてきた.しかし,片麻痺患者に対してどの指標がより起立動作の特徴を捉えているか,有用な指標を抽出するような研究は行われていない\cite{長田2012}.

本研究では起立動作を運動学的な観点と表面筋電図から計算される筋シナジー構造から評価する.運動学的な観点の指標として,片麻痺患者の関節角度や重心(Center of Mass,以下CoM)の軌道を用いる.

起立動作のCoMについてはいくつかの報告がなされている.
\textcolor{red}{\cite{Mourey2000}と\cite{Yang2017}を引いて,高齢者や片麻痺患者は上体を深く曲げて起立することを記述}
%高齢者は筋肉量が低下しているため,起立の際には転倒の危険性が発生する.より安定した起立のために,若年者よりも上体を深く曲げることでCoMを足に近づけてから立ち上がる\cite{Mourey2000}.
%さらに,片麻痺患者になると健常な高齢者よりもCoMが足に近づいてから立ち上がることが分かっている\cite{Yang2017}.
\\

起立動作中の筋電図(Electromyogram,以下EMG)については以下のような報告がある.
\textcolor{red}{\cite{Silva2013}と\cite{Gross1998}を引いて,前脛骨筋の活動タイミングなどについて記述.}\\
%Silvaらは健常者と片麻痺患者を対象とし,ヒラメ筋と前脛骨筋のEMGを計測した\cite{Silva2013}.片麻痺患者において健常者よりも,麻痺と非麻痺の両側のヒラメ筋の活動タイミングが早く,麻痺側の前脛骨筋の活動タイミングが遅かったと報告している.
%また,Grossらは若年者と片麻痺患者を対象として前脛骨筋と大腿四頭筋のEMGを計測した\cite{Gross1998}.片麻痺患者において,健側の足の位置を変えることで前脛骨筋の大腿四頭筋の活動が増加することを報告した.

本研究ではヒトの運動のメカニズムを理解するべく,筋シナジー仮説を用いる.


\textcolor{red}{筋シナジー仮説とは何かを説明.}
\\

運動学的な指標や筋シナジー構造の詳細な算出方法や計測実験については第\ref{chap:investigation}章にて述べる.

%\clearpage
%%%%%%%%%%%%%%%%%%%%%%%%%%%%%%%%%%%%%%%%%%%%%%%%%%%%%%%%%%%%%%%%%%%%%%%%%%%%%%%%%%%%%%%%%%%%%%%%%%%%%%%%%%%%%%%%%%%%%%%%%%%%%%%%%%%
\subsection{理学療法士の技能解析}
\label{subsec:analysys_chap2}

片麻痺患者のリハビリテーション中において,理学療法士はどのような介入を行っているのかを調査することを前述した.本項では,理学療法士の技能の解析方法について説明する.\\

理学療法士の技能について,Tsusakaらは起立動作における理学療法士のスキル分析をしている\cite{Tsusaka2015}.
\textcolor{red}{スキル分析について記述.}\\

本研究では理学療法士の技能の調査方法として片麻痺患者に介入する腕の筋肉の表面筋電図を計測する.
理学療法士は図\ref{fig:view}に示すように,片麻痺患者の麻痺側の膝と骨盤後面に介入する.上肢の曲げ伸ばしをどのタイミングで行っているかを調査するため,理学療法士の腕の筋肉の表面筋電図を計測する.
\\

\begin{figure}[b]
	\begin{center}
		\includegraphics[width=8cm]{./Chap2/fig/view.eps}
		\caption{理学療法士による介入の様子}
		\label{fig:view}
	\end{center}
\end{figure}

理学療法士の技能の詳細な調査方法は第\ref{chap:analysys}章にて述べる.

%\clearpage

%%%%%%%%%%%%%%%%%%%%%%%%%%%%%%%%%%%%%%%%%%%%%%%%%%%%%%%%%%%%%%%%%%%%%%%%%%%%%%%%%%%%%%%%%%%%%%%%%%%%%%%%%%%%%%%%%%%%%%%%%%%%%%%%%%%
\subsection{起立動作の支援機器の開発}
\label{subsec:development_chap2}

本研究では理学療法士の技能を活用した,起立動作の支援機器の開発について前述した.
本項では,技能を活用した支援機器の従来研究について説明する.\\

\textcolor{red}{看護師の技能を活かした起立動作の支援システムを引く\cite{Chugo2007}.
	理学療法士のスキル活かした起立アシストロボットの開発を引く\cite{津坂2017}.
}\\



起立動作の支援機器の開発についての詳細な方法は第\ref{chap:development}章にて述べる.

\clearpage

%%%%%%%%%%%%%%%%%%%%%%%%%%%%%%%%%%%%%%%%%%%%%%%%%%%%%%%%%%%%%%%%%%%%%%%%%%%%%%%%%%%%%%%%%%%%%%%%%%%%%%%%%%%%%%%%%%%%%%%%%%%%%%%%%%%
\section{おわりに}
\label{sec:outro_chap2}

本章では,本研究のアプローチについて述べた.

%まず\ref{sec:problem_setteing_chap2}節にて,本研究において解決するべき課題について述べた.

\ref{sec:approach_chap2}節にて,本手法のアプローチと解決すべき課題について述べた.
\ref{subsec:investigate_chap2}節にて,理学療法士によるリハビリ中の片麻痺患者の起立動作の変化を調査するアプローチについて述べた.
\ref{subsec:analysys_chap2}項にて,理学療法士の技能を解析するためのアプローチについて述べた.
\ref{subsec:development_chap2}項にて,支援機器を開発するアプローチについて述べた.\\

次章では,技能解析について詳しく説明する.

\clearpage

%%%%%%%%%%%%%%%%%%%%%%%%%%%%%%%%%%%%%%%%%%%%%%%%%%%%%%%%%%%%%%%%%%%%%%%%%%%%%%%%%%%%%%%%%%%%%%%%%%%%%%%%%%%%%%%%%%%%%%%%%%%%%%%%%%%
%%% Local Variables:
%%% mode: katex
%%% TeX-master: "../thesis"
%%% End:
