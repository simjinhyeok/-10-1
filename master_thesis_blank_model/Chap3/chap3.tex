\chapter{リハビリテーションによる片麻痺患者への影響の調査}
\label{chap:investigation}
\minitoc

\thispagestyle{empty}

\newpage
%%%%%%%%%%%%%%%%%%%%%%%%%%%%%%%%%%%%%%%%%%%%%%%%%%%%%%%%%%%%%%%%%%%%%%%%%%%%%%%%%%%%%%%%%%%%%%%%%%%%%%%%%%%%%%%%%%%%%%%%%%%%%%%%%%%
\section{はじめに}
\label{sec:intro_chap3}

本章では,理学療法士のリハビリテーションによる片麻痺患者への影響について述べる.

\ref{sec:method_chap3}節にて,片麻痺患者への影響の調査方法の概要について述べる.本研究では片麻痺患者の重心軌道,関節角度といった運動学的な手法や筋シナジー構造による評価を評価手法として採用する.\ref{subsec:method_kinematics_chap3}項にて,重心軌道や関節角度の算出方法について述べる.\ref{subsec:method_synergy_chap3}項にて,筋シナジー構造について述べる.

\ref{sec:experiment_chap3}節にて,片麻痺患者に対して行った計測実験について述べる.\ref{subsec:subject_chap3}項にて,対象とした片麻痺患者の属性について述べる.\ref{subsec:device_chap3}項にて,計測に用いた装置について述べる.\ref{subsec:method_chap3}項にて,計測実験の手順について述べる.

\ref{sec:result_chap3}節にて,計測実験の結果について述べる.\ref{subsec:result_kinematics_chap3}項にて,重心軌道や関節角度の調査結果について述べる.\ref{subsec:result_synergy_chap3}項にて,筋シナジー構造の調査結果について述べる.

\ref{sec:discussion_chap3}節にて,計測実験の結果の考察について述べる.\ref{subsec:discussion_kinematics_chap3}項にて,重心軌道や関節角度の調査結果の考察について述べる.\ref{subsec:discussion_synergy_chap3}項にて,筋シナジー構造の調査結果の考察について述べる.

\ref{sec:outro_chap3}節にて,本章のまとめを述べる.

\textcolor{red}{改ページする予定です.}
%\clearpage

%%%%%%%%%%%%%%%%%%%%%%%%%%%%%%%%%%%%%%%%%%%%%%%%%%%%%%%%%%%%%%%%%%%%%%%%%%%%%%%%%%%%%%%%%%%%%%%%%%%%%%%%%%%%%%%%%%%%%%%%%%%%%%%%%%%
%%%%%%%%%%%%%%%%%%%%%%%%%%%%%%%%%%%%%%%%%%%%%%%%%%%%%%%%%%%%%%%%%%%%%%%%%%%%%%%%%%%%%%%%%%%%%%%%%%%%%%%%%%%%%%%%%%%%%%%%%%%%%%%%%%%
\section{片麻痺患者への影響の調査方法}
\label{sec:method_chap3}
\ref{subsec:investigate_chap2}項にて,であることを述べた.本節では,理学療法士のリハビリテーションによる片麻痺患者への影響の調査方法の概要について述べる.\\

片麻痺患者のリハビリにおける起立動作の調査方法の概要を図\ref{fig:approach_hemi}に示す.本研究ではモーションキャプチャシステム,無線筋電計,床反力計を用いて片麻痺患者の運動を計測する.

モーションキャプチャシステムでは患者に貼り付けたモーションキャプチャ用のマーカの位置を計測し,関節角度や重心(Center of Mass,以下CoM)軌道が算出可能である.
算出方法は\ref{subsec:method_kinematics_chap3}項にて,述べる.

無線筋電計では患者に貼り付ることで表面筋電図(Electromyogram,以下EMG)が計測可能である.
無線筋電計を用いて筋シナジー構造を算出する方法を\ref{subsec:method_synergy_chap3}項にて述べる.\\

%\begin{figure}[b]
%	\begin{center}
%		\includegraphics[width=8cm]{./Chap3/fig/approach_hemi.eps}
%		\caption{片麻痺患者の起立動作の調査方法の概要}
%		\label{fig:approach_hemi}
%	\end{center}
%\end{figure}

\subsection{運動学的による解析方法}
\label{subsec:method_kinematics_chap3}
本研究では運動学的な解析手法として重心軌道や関節角度を算出する方法を採用する.\textcolor{red}{計算方法を述べる.}\\

\subsection{筋シナジー構造による解析方法}
\label{subsec:method_synergy_chap3}
本研究では筋シナジー構造を解析することで,リハビリテーションによる片麻痺患者への影響を調査する.\textcolor{red}{計算方法を述べる.}\\

\textcolor{red}{改ページする予定です.}
%\clearpage
%%%%%%%%%%%%%%%%%%%%%%%%%%%%%%%%%%%%%%%%%%%%%%%%%%%%%%%%%%%%%%%%%%%%%%%%%%%%%%%%%%%%%%%%%%%%%%%%%%%%%%%%%%%%%%%%%%%%%%%%%%%%%%%%%%%
%%%%%%%%%%%%%%%%%%%%%%%%%%%%%%%%%%%%%%%%%%%%%%%%%%%%%%%%%%%%%%%%%%%%%%%%%%%%%%%%%%%%%%%%%%%%%%%%%%%%%%%%%%%%%%%%%%%%%%%%%%%%%%%%%%%
\section{片麻痺患者の計測実験}
\label{sec:experiment_chap3}
本章では片麻痺患者へ行った実験について述べる.下記に述べる実験は森之宮病院倫理委員会の承認を受け,実施された.

%%%%%%%%%%%%%%%%%%%%%
\subsection{被験者}
\label{subsec:subject_chap3}
対象とした片麻痺患者の属性について述べる.

%%%%%%%%%%%%%%%%%%%%%
\subsection{計測装置}
\label{subsec:device_chap3}
計測に用いた装置について述べる.

%%%%%%%%%%%%%%%%%%%%%
\subsection{実験手順}
\label{subsec:method_chap3}
計測実験の手順について述べる.

\textcolor{red}{改ページする予定です.}
%\clearpage
%%%%%%%%%%%%%%%%%%%%%%%%%%%%%%%%%%%%%%%%%%%%%%%%%%%%%%%%%%%%%%%%%%%%%%%%%%%%%%%%%%%%%%%%%%%%%%%%%%%%%%%%%%%%%%%%%%%%%%%%%%%%%%%%%%%
%%%%%%%%%%%%%%%%%%%%%%%%%%%%%%%%%%%%%%%%%%%%%%%%%%%%%%%%%%%%%%%%%%%%%%%%%%%%%%%%%%%%%%%%%%%%%%%%%%%%%%%%%%%%%%%%%%%%%%%%%%%%%%%%%%%
\section{実験結果}
\label{sec:result_chap3}

%%%%%%%%%%%%%%%%%%%%%
\subsection{運動学的な調査の結果}
\label{subsec:result_kinematics_chap3}
重心軌道や関節角度の調査結果について述べる.

%%%%%%%%%%%%%%%%%%%%%
\subsection{筋シナジー構造の調査の結果}
\label{subsec:result_synergy_chap3}
筋シナジー構造の調査結果について述べる.

\textcolor{red}{改ページする予定です.}
%\clearpage
%%%%%%%%%%%%%%%%%%%%%%%%%%%%%%%%%%%%%%%%%%%%%%%%%%%%%%%%%%%%%%%%%%%%%%%%%%%%%%%%%%%%%%%%%%%%%%%%%%%%%%%%%%%%%%%%%%%%%%%%%%%%%%%%%%%
%%%%%%%%%%%%%%%%%%%%%%%%%%%%%%%%%%%%%%%%%%%%%%%%%%%%%%%%%%%%%%%%%%%%%%%%%%%%%%%%%%%%%%%%%%%%%%%%%%%%%%%%%%%%%%%%%%%%%%%%%%%%%%%%%%%
\section{考察}
\label{sec:discussion_chap3}

%%%%%%%%%%%%%%%%%%%%%
\subsection{運動学的な調査の結果}
\label{subsec:discussion_kinematics_chap3}
重心軌道や関節角度の考察について述べる.

%%%%%%%%%%%%%%%%%%%%%
\subsection{筋シナジー構造の考察}
\label{subsec:discussion_synergy_chap3}
筋シナジー構造の考察について述べる.

\textcolor{red}{改ページする予定です.}
%\clearpage
%%%%%%%%%%%%%%%%%%%%%%%%%%%%%%%%%%%%%%%%%%%%%%%%%%%%%%%%%%%%%%%%%%%%%%%%%%%%%%%%%%%%%%%%%%%%%%%%%%%%%%%%%%%%%%%%%%%%%%%%%%%%%%%%%%%
\section{おわりに}
\label{sec:outro_chap3}
本章では,片麻痺患者へ行った実験について述べた.

\ref{sec:method_chap3}節にて,片麻痺患者への影響の調査方法の概要について述べた.本研究では片麻痺患者の重心軌道,関節角度といった運動学的な手法や筋シナジー構造による評価を評価手法として採用した.\ref{subsec:method_kinematics_chap3}項にて,重心軌道や関節角度の算出方法について述べた.\ref{subsec:method_synergy_chap3}項にて,筋シナジー構造について述べた.

\ref{sec:experiment_chap3}節にて,片麻痺患者に対して行った計測実験について述べた.\ref{subsec:subject_chap3}項にて,対象とした片麻痺患者の属性について述べた.\ref{subsec:device_chap3}項にて,計測に用いた装置について述べた.\ref{subsec:method_chap3}項にて,計測実験の手順について述べた.

\ref{sec:result_chap3}節にて,計測実験の結果について述べた.\ref{subsec:result_kinematics_chap3}項にて,重心軌道や関節角度の調査結果について述べた.\ref{subsec:result_synergy_chap3}項にて,筋シナジー構造の調査結果について述べた.

\ref{sec:discussion_chap3}節にて,計測実験の結果の考察について述べた.\ref{subsec:discussion_kinematics_chap3}項にて,重心軌道や関節角度の調査結果の考察について述べた.\ref{subsec:discussion_synergy_chap3}項にて,筋シナジー構造の調査結果の考察について述べた.\\

次章では,理学療法士の技能の解析について説明する.

\clearpage

%%%%%%%%%%%%%%%%%%%%%%%%%%%%%%%%%%%%%%%%%%%%%%%%%%%%%%%%%%%%%%%%%%%%%%%%%%%%%%%%%%%%%%%%%%%%%%%%%%%%%%%%%%%%%%%%%%%%%%%%%%%%%%%%%%%
%%% Local Variables:
%%% mode: katex
%%% TeX-master: "../thesis"
%%% End:
