\chapter{結論}
\thispagestyle{empty}
\label{chap6}
\minitoc

\newpage
%%%%%%%%%%%%%%%%%%%%%%%%%%%%%%%%%%%%%%%%%%%%%%%%%%%%%%%%%%%%%%%%%%%%%%%%%%%%%%%
%==============================================================================
%まとめ
%==============================================================================
\section{結論}

本論文では,直線情報に基づく全天球カメラの高速な6自由度の位置姿勢推定手法を提案した.
\vspace{\baselineskip}

第1章にて,研究背景として位置姿勢推定における全天球カメラの優位性を示し,関連する従来研究では,大域的アプローチによる全天球カメラの6自由度の位置姿勢推定として,リアルタイムのアプリケーションに適用できるような高速な手法は提案されていないことを述べた.これに基づき本研究の目的を「全天球カメラを用いた大域的なアプローチによる高速な6自由度の位置姿勢推定」と設定した.
\vspace{\baselineskip}

第2章において,本研究のアプローチについて述べた.
本研究では,位置姿勢推定において位置の推定と姿勢の推定を分離して行い,初めに姿勢を推定してから位置を推定することを述べた.
姿勢推定は消失点を用いて行い,画像の勾配のみから推定することで高速化することを述べた.
位置推定は,$xy$座標の推定と$z$座標の推定を分離してそれぞれを高速に行うことを述べた.
\vspace{\baselineskip}

第3章では姿勢推定の提案手法について詳しく説明した.
姿勢推定は,単位球面勾配ベクトルの分布とマンハッタンワールドにおける互いに直交する3平面を比較することによって行うことを説明した.
単位球面勾配ベクトルの計算は,画像内の2次元の勾配ベクトルを変換することによって行い,歪みによる勾配の計算精度の低下を抑えるために,歪みの大きい領域における勾配の計算には,入力画像を球面に変換して3次元空間において回転させた画像を用いることを述べた.
\vspace{\baselineskip}

第4章では位置推定の提案手法について詳しく説明した.
$z$軸方向に伸びる直線が投影された球上の直線と全天球カメラの$xy$平面の交点の分布は,カメラの$xy$座標にのみ依存することから,これを利用してまず$xy$座標を推定することを述べた.
その後,$x$軸方向に伸びる直線が投影された球上の直線と全天球カメラの$yz$平面の交点の分布を利用して$z$座標を推定する手法を提案した.
\vspace{\baselineskip}

第5章では,提案手法の有効性を示すために行った実験について述べた.
シミュレーション環境および実環境において位置姿勢推定実験を行い,高速に推定可能であることを実証した.
一方で,推定精度に関しては改善の必要があることが確認された.
\vspace{\baselineskip}

本論文により,直線情報に基づく全天球カメラの高速な6自由度の位置姿勢推定手法が確立された.

\newpage
%==============================================================================
%今後の展望
%==============================================================================
\section{今後の展望}

本論文では,直線情報に基づく全天球カメラの高速な6自由度の位置姿勢推定手法を構築した.
位置姿勢を高速に推定することができたものの,推定精度に関しては課題が残る結果となった.
さらなる性能の向上のためには,直線以外の情報の活用が考えられる.
例えば,直線周辺のテクスチャ情報を利用することなどが例として挙げられる.
これにより,精度やロバスト性の向上が期待される.
\vspace{\baselineskip}

%また,本手法は大域的アプローチによって全天球カメラの位置姿勢推定を行った.環境全域からある程度の位置を推定することができれば全天球カメラ画像と3次元モデルの直線の対応を推定することができる.直線の対応を推定できれば,直線の対応を用いて位置姿勢を推定することでより高精度に位置姿勢推定が可能となる.
%\vspace{\baselineskip}

また,本手法で使用した変数や閾値の中には試行錯誤的に定められたものが数点存在する.
環境によっては,これらのパラメータが精度に影響を与えることも考えられる.
特に,エッジ検出のパラメータは直線検出の精度に大きな影響を与える可能性があり,環境によって適切に定める必要がある.
従って,これらのパラメータを適切に決定する手法の構築は本手法の実用化において必要な課題である.
\vspace{\baselineskip}

%また,本手法では屋内環境に着目した手法を提案したが,様々な用途に対応するためには屋外環境にも適用可能な手法の構築が必要となる.
%屋外環境では,日照条件の違いによる照明の変化や車などの移動物体など,屋内環境にはない課題が考えられる.
%これらの課題を克服するために,屋外環境に対応可能なディスクリプタの構築が必要となる.
%\vspace{\baselineskip}

このような課題を解決することで,本提案手法は全天球カメラの位置姿勢推定を行う際の一般的枠組みとして幅広く応用可能になることが期待できる.


\newpage

%%%%%%%%%%%%%%%%%%%%%%%%%%%%%%%%%%%%%%%%%%%%%%%%%%%%%%%%%%%%%%%%%%%%%%%%%%%%%%%%
%%%% Local Variables:
%%%% mode: katex
%%%% TeX-master: "../thesis"
%%%% End:
