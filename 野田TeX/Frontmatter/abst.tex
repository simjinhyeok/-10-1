\chapter*{概要}
\markboth{概要}{}
\label{abst}
\def\thepage{}
\thispagestyle{empty}

人間が直接作業することの困難な環境では,ロボットを用いて作業を行うことが期待されている.このとき,作業位置を特定するなどの目的でロボットの運動を推定することが求められる.例えば,ロボットを用いた橋梁の点検では,点検箇所を把握することが必要であり,画像情報を用いたロボットの運動推定に関する研究が行われている[1].橋梁点検などのように,作業対象物に近接した状態においてより多くの情報を取得するためには,視野の限られた通常のカメラよりも視野の広いカメラが適している.本研究では,周囲360◦ の情報を一度に取得できる極めて広い視野を有する全天球カメラに注目する.\\
全天球カメラを用いた運動推定の研究として,Structurefrom Motion(SfM)による手法が提案されている[2].SfM は,カメラ1 台の移動によって異なる視点からの画像を取得し,3次元復元と運動推定を行う手法である[3].橋梁の点検においては,点検箇所を把握するために絶対的な移動距離を推定することが求められる.しかしSfM には,スケールの不定性や,カメラの移動量が小さい場合に運動推定精度が低下するという問題がある.\\
その他の運動推定の研究として,複数台のカメラを用いたステレオカメラによる3次元計測の結果を利用する運動推定の手法が提案されている[4].基線長が固定されているステレオカメラによる運動推定の性質として,上述のスケールの不定性や,カメラの移動量が小さい場合の運動推定精度の低下という問題は生じない.しかし,[4] は通常のカメラを用いており,全天球カメラの広い視野を活用したステレオによる運動推定の手法は構築されていない.\\
視野の広いカメラを複数台用いた3次元計測の研究としては,複数台の魚眼カメラを用いた手法が提案されている[5].しかし,[5] は対応点を曲線上で探索しており,また180◦ 程度の範囲のみを取得する魚眼ステレオカメラを用いて運動推定を行う場合,移動前後での計測に共通部分が含まれずに運動が推定できない場合が生じうるという問題がある.全天球カメラを用いた場合には,対応点の探索を直線上で行うことができ,またカメラがどのような移動をしても必ず共通部分が得られて運動を推定することができる.\\
本研究では,全天球ステレオカメラを用いた3次元計測結果を利用する運動推定の手法の構築を目的とする.ステレオカメラとの幾何的な関係から計測における誤差が大きいような点を,信頼度が低いとして除去することで運動推定の精度を向上させる手法を提案する.

本論文では全天球ステレオカメラの運動推定手法を構築した.
まず提案手法において基本となる2視点の画像を用いた手法について述べた.
この手法では,カメラの前に配置する透明平板の姿勢は任意とした.
しかし,この手法において透明平板をカメラの光軸と垂直に配置した場合,
特殊なケースとして扱えることが明らかとなり,その特殊性を利用して改良し,
透明平板がカメラ光軸に垂直な場合の手法として確立した.
これは平板の姿勢を任意とした場合の手法に比べ,必要な対応点数が1つ減るという特徴がある.
上述の手法はいずれも2視点間の関係のみを用いた手法であったため,
最後に前述の手法を多視点に拡張し,かつ得られた3次元情報を最適化するためのバンドル調整を用いた手法について述べた.
提案手法では屈折が発生しているため,この効果を考慮した新しい誤差関数を定義した.
なお,このバンドル調整には初期値が必要であり,前述の2視点でスケール復元が可能な手法により得られた値を初期値とした.
これにより,2視点だけでなく多視点の画像を用いて最適化が可能となり,かつ最適化処理を入れることでよりロバスト性の高い手法となった.

次に提案手法の有効性を示すため,実験を行った.
その結果,提案手法は対応点座標値の誤差に影響を受けやすいことが分かった.しかし,バンドル調整を用いた手法において,
視点の数を増やすことでこの誤差に対するロバスト性が向上することが示された.一方で,透明平板を厚くすること,解像度の高いカメラを用いることが
精度の高い計測を行うために重要であることも明らかとなった.
実画像を用いた実測実験では,実際にカメラの前に透明平板を配置した状態で撮影した画像に対して
提案手法を適用した.その結果高い精度で計測対象のスケールを復元することができ,
提案手法が有効であることが示された.

\thispagestyle{empty}

\newpage
%%%%%%%%%%%%%%%%%%%%%%%%%%%%%%%%%%%%%%%%%%%%%%%%%%%%%%%%%%%%%%%%%%%%%%%%%%%%%%%

%%%%%%%%%%%%%%%%%%%%%%%%%%%%%%%%%%%%%%%%%%%%%%%%%%%%%%%%%%%%%%%%%%%%%%%%%%%%%%%
%%% Local Variables:
%%% mode: katex
%%% TeX-master: "../thesis"
%%% End:
