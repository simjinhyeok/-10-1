\chapter{従来のStructure from Motion手法}
\lhead[付録A 従来のStructure from Motion手法]{}
\setcounter{page}{1}
\renewcommand{\thepage}{A--\arabic{page}}

\thispagestyle{empty}

\newpage
%%%%%%%%%%%%%%%%%%%%%%%%%%%%%%%%%%%%%%%%%%%%%%%%%%%%%%%%%%%%%%%%%%%%%%%%%%%%%%%
\section{はじめに}

本付録では,屈折を用いていない従来のStructure from Motionについて述べる.

A.2節にて,従来のStructure from Motionの概要について述べる.

A.3節にて,従来のStructure from Motionの理論の詳細について述べる.



\newpage

\section{従来のStructure from Motionの概要}

ここでは,屈折を用いていない従来のStructure from Motionの概要を述べる.
本付録では,従来のStructure from Motionの計算方法の1つである8点法について述べる.
8点法はLonguet-HigginsやHartleyらによって開発された,Structure from Motionの計算方法の中で最も基本的な手法である\cite{Longuet-Higgins1981}\cite{Hartley2004}.

従来の8点法の流れを\mbox{図\ref{}に示す.}


\newpage

%%%%%%%%%%%%%%%%%%%%%%%%%%%%%%%%%%%%%%%%%%%%%%%%%%%%%%%%%%%%%%%%%%%%%%%%%%%%%%%
%%% Local Variables:
%%% mode: katex
%%% TeX-master: "../thesis"
%%% End:
