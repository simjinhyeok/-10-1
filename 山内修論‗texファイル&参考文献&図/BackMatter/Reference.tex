\chapter*{参考文献}
%\addcontentsline{toc}{chapter}{参考文献}
\lhead[参考文献]{}
%\renewcommand{\refname}{引用文献}
\thispagestyle{empty}

\newpage

%==============================================================================
%和文文献
%==============================================================================
\subsection*{\textmc{<和文文献>}}
\begin{mythebibliography}{}

% い
\bibitem[生田 2005]{生田2005} % スペクトル画像は広い面積を撮影できる
\leavevmode \\生田~英轄, 宮野~道雄, 糸井川~栄一,西村~明儒,田中~裕,梶原~浩一,熊谷~良雄:
\newblock ``統合データベースに基づく兵庫県南部地震による人的被害の発生機構に関する分析'',
\newblock 日本建築学会計画系論文集,No.~590,pp.~71--78, 2006.
\\

\bibitem[岩崎 1980]{岩崎1980} % 地盤液状化指数
\leavevmode \\岩崎~敏男, 龍岡~文夫, 常田~賢一, 安田~進:
\newblock ``地震時地盤液状化の程度の予測について'',
\newblock 土と基礎, Vol.~28, No.~4, pp.~23--29, 1980.
\\

% お
\bibitem[太田 1988]{太田1988} % 土質パラメータ
\leavevmode \\太田~秀樹, 鍋谷~雅司, 藤井~信二, 山本~松生:
\newblock ``弾・粘塑性有限要素解析の入力パラメーター決定における一軸圧縮強度の利用'',
\newblock 土木学会論文集, Vol.~1988, No.~400, pp.~45--54, 1988.
\\

\bibitem[小川 1989]{小川1989} % ハイパースペクトル画像を用いた土の種類の識別
\leavevmode \\小川~茂男, 安養~寺久男, 福本~昌人:
\newblock ``分光反射率による土壌の腐植量および土壌水分の推定'',
\newblock 農業土木学会誌,Vol.~57,No.~6,pp.~465--469, 1989.
\\

\bibitem[小田 1971]{小田1971} % 土の土質力学的性質に影響を与える基礎的な性質
\leavevmode \\小田~匡寛, 榎本~文勇, 鈴木~正:
\newblock ``砂粒子の形状・組成が砂の土質工学的性質に及ぼす影響に関する研究'',
\newblock 土と基礎, Vol.~19, No.~2, pp.~5--12, 1971.
\\

% か
\bibitem[角田 2013]{角田2013} % 酸素原子と水素原子の間の振動収縮と変角運動のために,吸光波長帯は大きい幅で温度依存性あり
\leavevmode \\角田~直人, 近藤~克哉, 有本~英伸, 山田~幸生:
\newblock ``水の近赤外吸収特性を利用した非接触温度イメージング'',
\newblock システム制御情報学会誌,Vol.~57, No.~12, pp.~493--498, 2013.
\\

% き
\bibitem[気象キャスターネットワーク 2014]{気象キャスターネットワーク2014} % 日本に上陸する台風の数とその世界の中の割合
\leavevmode \\気象キャスターネットワーク:
\newblock ``数字で観た世界の台風'', 
\newblock \url{http://www.fudeyasu.ynu.ac.jp/education/lec/wnc/world-v1.pdf}, 2014, 
\newblock \mbox{閲覧日 2020.1.10}.
\\

\newpage

\bibitem[気象庁 2018]{気象庁2018} % 土砂災害の発生件数は記述されていない
\leavevmode \\気象庁:
\newblock ``平成30年7月豪雨(前線及び台風第7号による大雨等)'', 
\newblock \hspace{-1mm} \url{https://www.data.jma.go.jp/obd/stats/data/bosai/report/2018/20180713/20180713.html}, 2018, 
\newblock \mbox{閲覧日 2019.12.9}.
\\

\bibitem[気象庁 2019a]{気象庁2019a} % 現在の日本の活火山の数
\leavevmode \\気象庁:
\newblock ``活火山とは'', 
\newblock \url{https://www.data.jma.go.jp/svd/vois/data/tokyo/STOCK/kaisetsu/katsukazan_toha/katsukazan_toha.html}, 2019, 
\newblock \mbox{閲覧日 2019.12.22}.
\\

\bibitem[気象庁 2019b]{気象庁2019b} % 日本に接近する台風の数の平均値
\leavevmode \\気象庁:
\newblock ``台風の平年値'', 
\newblock \url{https://www.data.jma.go.jp/fcd/yoho/typhoon/statistics/average/average.html}, 2019, 
\newblock \mbox{閲覧日 2019.12.10}.
\\

% こ
\bibitem[国交省 2002]{国交省2002} % ガレキの下敷きになった被災者の救助は3日以内に実施
\leavevmode \\国土交通省:
\newblock ``阪神・淡路大震災の経験に学ぶ'', \\
\newblock \url{https://www.kkr.mlit.go.jp/plan/daishinsai/}, 2002, 
\newblock \mbox{閲覧日 2019.12.10}.
\\

\bibitem[国交省 2004]{国交省2004} % 日本の年平均降水量と世界の年平均降水量
\leavevmode \\国土交通省:
\newblock ``水害対策を考える'', 
\newblock \url{https://www.mlit.go.jp/river/pamphlet_jirei/bousai/saigai/kiroku/suigai/suigai_3-1-1.html}, 2004, 
\newblock \mbox{閲覧日 2019.12.10}.
\\

\bibitem[国交省 2007]{国交省2007} % 地震,火山,豪雨が土砂災害を誘発
\leavevmode \\国土交通省:
\newblock ``土砂災害と対策の概要'', \\
\newblock \url{http://www.mlit.go.jp/common/001024560.pdf}, 2007, 
\newblock \mbox{閲覧日 2019.12.9}.
\\

\bibitem[国交省 2016]{国交省2016} % 土砂災害発生後には迅速な復旧工事が必要
\leavevmode \\国土交通省:
\newblock ``大規模災害時における市町村対応の現状と課題'', 
\newblock \url{https://www.mlit.go.jp/river/shinngikai_blog/shityosonshien/dai01kai/pdf/2_genjotokadai.pdf}, 2016, 
\newblock \mbox{閲覧日 2019.12.11}.
\\

\bibitem[国交省 2019]{国交省2019} % 北海道胆振東部地震,2018年の土砂災害発生件数・被災者数・被害戸数
\leavevmode \\国土交通省:
\newblock ``平成30年の土砂災害についてとりまとめました'', \\
\newblock \url{https://www.mlit.go.jp/common/001287382.pdf}, 2019, 
\newblock \mbox{閲覧日 2019.12.9}.
\\

\bibitem[小嶋 1996]{小嶋1996} % 粒度分布の赤外スペクトルへの影響
\leavevmode \\小嶋~純:
\newblock ``赤外分光分析による石英粉じんの粒度分布推計'',
\newblock 分析化学, Vol.~45, No.~11, pp.~999--1004, 1996.
\\

% さ
\bibitem[三枝 1991]{三枝1991} % pH値と植物の育ちやすさの関係,さいぐさ
\leavevmode \\三枝~正彦:
\newblock ``低pH土壌における作物の生育'',
\newblock 日本土壌肥料学雑誌, Vol.~62, No.~4, pp.~451--459, 1991.
\\

% す
\bibitem[図子 1993]{図子1993} % pH値と植物の育ちやすさの関係
\leavevmode \\図子~光太郎,  生原~喜久雄, 相場~芳憲, 小林~健吾:
\newblock ``森林土壌の交換性イオンの特性が土壌溶液のイオンの動態に及ぼす影響'',
\newblock 日本林学会誌, Vol.~75, No.~3, pp.~176--184, 1993.
\\

% た
\bibitem[高田 1983]{高田1983} % 粘性土は圧密により深いところにあるほど固くなる
\leavevmode \\高田~直俊:
\newblock ``軟弱粘土の自重圧密過程の数値解析'',
\newblock 土木学会論文報告集,Vol.~1983,No.~334,pp.~113--121, 1983.
\\

\bibitem[田代 2013]{田代2013} % スペクトル画像は広い面積を撮影できる
\leavevmode \\田代~智子, 仲村~匡司:
\newblock ``イメージング分光法による材色分布の特徴抽出'',
\newblock 材料,Vol.~62,No.~4,pp.~248--253, 2013.
\\

\bibitem[龍岡 1980]{龍岡1980} % 地盤液状化指数
\leavevmode \\龍岡~文夫:
\newblock ``地震時における地盤の液状化の激しさの程度の予測'',
\newblock 	生産研究, Vol.~32, No.~1, pp.~2--10, 1980.
\\

\bibitem[田畑 2006]{田畑2006} % スペクトル画像は広い面積を撮影できる
\leavevmode \\田畑~直樹, 岡田~成幸:
\newblock ``地震時の木造建築物倒壊に伴う死者数推定に向けた棟死亡率関数の提案'',
\newblock 日本建築学会構造系論文集,No.~605,pp.~117--123, 2005.
\\

\bibitem[玉手 2008]{玉手2008} % 軟弱な地盤を原因とする建設機械の転倒に関する研究
\leavevmode \\玉手~聡:
\newblock ``基礎工事用大型建設機械の転倒防止に関する研究'',
\newblock 厚生労働科学研究費補助金(労働安全衛生総合研究事業)平成19年度総括分担報告書, pp.~3--31, 2008.
\\

\bibitem[玉手 2014]{玉手2014} % 軟弱な地盤を原因とする建設機械の転倒に関する研究
\leavevmode \\玉手~聡, 堀~智仁, 前田~豊:
\newblock ``3308 移動式クレーンの安定確保に必要な地耐力の検討'',
\newblock 日本機械学会 第23回交通・物流部門大会講演論文集(TRANSLOG2014), OS5-2, pp.~169--172, 東京, December 2014.
\\

% つ
\bibitem[蔦 2002]{蔦2002} % スペクトル画像は広い面積を撮影できる
\leavevmode \\蔦~瑞樹, 杉山~純一, 相良~泰行:
\newblock ``ハイパースペクトルシステムによる近赤外分光イメージング手法'',
\newblock 映像メディア学会誌,Vol.~56,No.~12,pp.~2037--2040, 2002.
\\

% と
\bibitem[豊田 2008]{豊田2008} % 災害によるライフラインの停止に伴う経済的被害
\leavevmode \\豊田~安由美, 庄司~学:
\newblock ``ライフライン事業者が想定する地震時応急復旧活動のシナリオとその相互依存関係'',
\newblock 地域安全学会論文集, Vol.~10, pp.~55--65, 2008.
\\

\bibitem[豊田 2010]{豊田2010} % 災害によるライフラインの停止に伴う経済的被害
\leavevmode \\豊田~安由美, 庄司~学:
\newblock ``首都圏に位置する電力・都市ガス・通信システムの地震時応急復旧活動に関する広域応援と道路交通支障の関係'',
\newblock 土木学会論文集A1(構造・地震工学), Vol.~66, No.~1, pp.~317--327, 2010.
\\

% な
\bibitem[内閣府 2010]{内閣府2010} % 日本で発生した地震の割合,
\leavevmode \\内閣府:
\newblock ``平成22年度版 防災白書'', 
\newblock \url{http://www.bousai.go.jp/kaigirep/hakusho/h22/bousai2010/html/honbun/2b_1s_1_01.htm}, 2010, 
\newblock \mbox{閲覧日 2019.12.9}.
\\

\bibitem[中島 2015]{中島2015} % 酸素原子と水素原子の間の振動収縮と変角運動
\leavevmode \\中島~利郎, 的場~修:
\newblock ``近赤外光の吸光特性を利用した水の相状態(液相,固相)の検出'',
\newblock 第57回自動制御連合講演会,2C11--3, pp.~1822--1823, 群馬, 2014.
\\

\bibitem[長田 2004]{長田2004} % スペクトル画像は広い面積を撮影できる
\leavevmode \\長田~典子, 真鍋~佳嗣, 井口~征士:
\newblock ``スペクトル画像計測とその応用'',
\newblock 電気学会論文誌C,Vol.~124,No.~6,pp.~1325--1331, 2004.
\\

\newpage

\bibitem[中野 1996]{中野1996} % スペクトル画像の解説
\leavevmode \\中野~恵一, 小宮~康宏:
\newblock ``マルチスペクトルカメラを用いた物体識別'',
\newblock 応用物理,Vol.~65,No.~5,pp.~496--499, 1996.
\\

% に
\bibitem[日本建設総合試験所 2019]{日本建設総合試験所2019} % 含水比の測定方法
\leavevmode \\一般財団法人 日本建設総合試験所:
\newblock ``日本建設総合試験所'', 
\newblock \url{https://www.gbrc.or.jp/assets/test_series/documents/so_02.pdf}, 2019, 
\newblock \mbox{閲覧日 2019.12.28}.
\\

% は
\bibitem[浜田 1986]{浜田1986} % 地盤液状化指数
\leavevmode \\浜田~政則, 安田~進, 磯山~龍二, 恵本~克利:
\newblock ``液状化による地盤の永久変位と地震被害に関する研究'',
\newblock 土木学会論文集, Vol.~1986, No.~376, pp.~221--229, 1986.
\\

% ひ
\bibitem[廣瀬 2014]{廣瀬2014} % 日本に上陸する台風の数とその世界の中の割合
\leavevmode \\廣瀬~駿:
\newblock ``全海域における台風の統計解析'', 
\newblock \url{http://www.fudeyasu.ynu.ac.jp/member/thesis/2013-hiroses/hirose2014.html}, 2014, 
\newblock \mbox{閲覧日 2020.1.10}.
\\

% ふ
\bibitem[古谷 2016]{古谷2016} % 遠隔操作ロボットを用いた無人でのコーン指数の測定
\leavevmode \\古屋~弘, 山田~祐樹, 栗生~暢雄, 清酒~芳夫, 森~直樹:
\newblock ``遠隔搭乗操作によるマルチクローラ型無人調査ロボットの開発'',
\newblock 大林組技術研究所報, No.~80, pp.~1--10, 2016.
\\

% ま
\bibitem[眞鍋 1996]{眞鍋1996} % スペクトル画像の最初の説明
\leavevmode \\眞鍋~佳嗣, 佐藤~宏介, 井口~征士:
\newblock ``物体認識のためのスペクトル画像による材質の判別'',
\newblock 電子情報通信学会論文誌 D, Vol.~J79--D2, No.~1, pp.~36--44, 1996.
\\

% み
\bibitem[三笠 1964]{三笠1964} % 土の応用的な性質と基礎的な性質の関係
\leavevmode \\三笠~正人:
\newblock ``土の工学的性質の分類表とその意義'',
\newblock 土と基礎, Vol.~12, No.~4, pp.~17--24, 1964.
\\

\bibitem[三隅 1992]{三隅1992} % 土質パラメータ
\leavevmode \\三隅~浩二:
\newblock ``正規圧密粘土の降伏曲線および弾塑性パラメータの決定'',
\newblock 土木学会論文集, Vol.~1992, No.~454, pp.~93--101, 1992.
\\

\newpage

% も
\bibitem[森本 1975]{森本1975} % 粘性土は圧密により深いところにあるほど固くなる
\leavevmode \\森本~幸裕, 増田~拓朗:
\newblock ``踏圧による土壌の圧密と樹木の生育状態について'',
\newblock 造園雑誌,Vol.~39,No.~2,pp.~34--42, 1975.
\\

% や
\bibitem[山口 1986]{山口1986} % 土質パラメータ
\leavevmode \\山口~晴幸, 松尾~啓, 大平~至徳, 木暮~敬二:
\newblock ``泥炭および泥炭地盤の土質工学的性質'',
\newblock 土木学会論文集, Vol.~1986, No.~370, pp.~271--280, 1986.
\\

% よ
\bibitem[横矢 2014]{横矢2014} % ハイパースペクトル画像の歴史と原理
\leavevmode \\横矢~直人, 岩崎晃:
\newblock ``ハイパースペクトル画像処理が拓く地球観測'',
\newblock 人工知能学会,Vol.~9,No.~4,pp.~357--365, 2014.
\\

% り
\bibitem[李 2000]{李2000} % 水は近赤外の波長帯の光を吸収
\leavevmode \\李~民賛, 笹~尾彰, 澁澤~栄, 酒井~憲司:
\newblock ``NIR反射スペクトルによる土壌パラメータの推定'',
\newblock 農業機械学会誌,Vol.~62,No.~3,pp.~111--120, 2000.
\\

% ろ
\bibitem[Robot Watch 2006]{RobotWatch2002} % 遠隔操作ロボットを用いた無人でのコーン指数の測定
\leavevmode \\Robot Watch:
\newblock ``テムザック、新型援竜「T-53」による土質調査実験を実施'', \\
\newblock \url{https://robot.watch.impress.co.jp/cda/news/2006/12/13/293.html}, 2006, 
\newblock \mbox{閲覧日 2019.7.22}.
\\

% わ
\bibitem[渡部 2007]{渡部2007} % 土質パラメータ
\leavevmode \\渡部~要一, 植田~智幸, 三枝~弘幸, 田中~政典, 菊池~喜昭:
\newblock ``性能設計概念に基づいた実用的土質定数設定法'',
\newblock 土木学会論文集C, Vol.~63, No.~2, pp.~553--565, 2007.
\\

\newpage


%==============================================================================
%英文文献
%==============================================================================
\subsection*{\textmc{\hspace{-1zw}<英文文献>}}

% A
\bibitem[Ayers 1982]{Ayers1982} % コーン指数を関数で算出
\leavevmode \\P.~D.~Ayers and J.~V.~Perumpral:
\newblock ``Moisture and Density Effect on Cone Index'',
\newblock {\it Transactions of the ASAE}, Vol.~25, No.~5, pp.~1169--1172, 1982.
\\

% B
\bibitem[Ben-Dor 1995]{Ben-Dor1995} % スペクトル画像を用いた土の性質の推定
\leavevmode \\E.~Ben-Dor and A.~Banin:
\newblock ``Near-Infrared Analysis as a Rapid Method to Simultaneously Evaluate Several Soil Properties'',
\newblock {\it Soil Science Society of America Journal}, Vol.~59, No.~2, pp.~364--372, 1995.
\\  

\bibitem[Ben-Dor 2002]{Ben-Dor2002} % スペクトル画像を用いた土の性質の推定
\leavevmode \\E.~Ben-Dor, K.~Patkin, A.~Banin and A.~Karnieli:
\newblock ``Mapping of Several Soil Properties Using DAIS-7915 Hyperspectral Scanner Data — a Case Study Over Clayey Soils in Israel'',
\newblock {\it International Journal of Remote Sensing}, Vol.~23, No.~6, pp.~1043--1062, 2002.
\\

% C
\bibitem[Chhaniyara 2012]{Chhaniyara2012} % 宇宙用のローバにペネトロメータをつけてコーン指数測定
\leavevmode \\S.~Chhaniyara, C.~Brunskill, B.~Yeomans, M.~C.~Matthews, C.~Saaj, S.~Ransom and L.~Richter:
\newblock ``Terrain~Trafficability~Analysis~and~Soil~Mechanical~Property~Identification~for~Planetary~Rovers:~A~Survey'',
\newblock {\it Journal~of~Terramechanics}, Vol.~49, pp.~115--128, 2012.
\\

\bibitem[Clark 2005]{Clark2005} % 熱帯雨林における異なる種類の植物の分布
\leavevmode \\M.~L.~Clark, D.~A.~Roberts and D.~B.~Clark:
\newblock ``Hyperspectral~Discrimination~of~Tropical~Rain~Forest~Tree~Species~at~Leaf~to~Crown~Scales'',
\newblock {\it Remote~Sensing~of~Environment}, Vol.~96, No.~3--4, pp.~375--398, 2005.
\\

\bibitem[Collins 1971]{Collins1971} % 走破性に影響を与える土の基礎的な性質
\leavevmode \\J.~G.~Collins:
\newblock ``Forecasting Trafficability of Soils; Relations of Strength to Othsr Properties of Fine-grained Soils and Sands with Fines'',
\newblock U. S. Army Engineer Waterways Experiment Station, Vicksburg, Technical Memorandum, No.~3--331, 1971.
\\

% E
\bibitem[Elbanna 1987]{Elbanna1987} % コーン指数を関数で算出
\leavevmode \\E.~B.~Elbanna and B.~D.~Witney:
\newblock ``Cone Pnetration Resistance Equation as a Function of the Clay Ratio, Soil Moisture Content and Specific Weight'',
\newblock {\it Journal of Terramechanics}, Vol.~24, No.~1, pp.~41--56, 1987.
\\

% F
% \bibitem[Fern\'andez 2015]{Fernandez2015} % 名前が省略されている論文と合わせる必要がある
% \leavevmode \\Roemi~Fern\'andez, He\'ctor~Montes, and Carlota~Salinas: 
% \newblock ``VIS-NIR,~SWIR~and~LWIR~Imagery~for~Estimation~of~Ground~Bearing~Capacity'',
% \newblock {\it Sensors}, Vol.~15, No.~6, pp.~13994--14015, 2015
% \\
\bibitem[Fern\'andez 2015]{Fernandez2015} % 近赤外線画像からコーン指数を推定
\leavevmode \\R.~Fern\'andez, H.~Montes and C.~Salinas: 
\newblock ``VIS-NIR,~SWIR~and~LWIR~Imagery~for~Estimation~of~Ground~Bearing~Capacity'',
\newblock {\it Sensors}, Vol.~15, No.~6, pp.~13994--14015, 2015.
\\

\bibitem[Flores 2014]{Flores2014} % 走破性に影響を与える土の基礎的な性質(含水比)
\leavevmode \\A.~N.~Flores, D.~Entekhabi and R.~L.~Bras:
\newblock ``Application of a Hillslope-scale Soil Moisture Data Assimilation System to Military Trafficability Assessment'',
\newblock {\it Journal of Terramechanics}, Vol.~51, pp.~53--66, 2014.
\\

% G
\bibitem[Goetz 1985]{Goetz1985} % スペクトル画像の解説
\leavevmode \\A.~F.~H.~Goetz, G.~Vane, J.~E.~Solomon, B.~N.~Rock:
\newblock ``Imaging Spectrometry for Earth Remote Sensing'',
\newblock {\it Science}, Vol.~228, No.~4704, pp.~1147--1153, 1985.
\\

% H
\bibitem[Haboudane 2004]{Haboudane2004} % 植物の育成の確認
\leavevmode \\D.~Haboudane, J.~R.~Miller, E.~Pattey, P.~J.~Zarco-Tejada and I.~B.~Strachan:
\newblock ``Hyperspectral Vegetation Indices and Novel Algorithms for Predicting Greeen LAI of Crop Canopies: Modeling and Validation in the Context of Precision Agriculture'',
\newblock {\it Remote Sensing of Environment}, Vol.~90, No.~3, pp.~337--352, 2004.
\\

\bibitem[Hendrick 1969]{Hendrick1969} % コーンペネトロメータの派生形
\leavevmode \\J.~G.~Hendrick:
\newblock ``Recording Soil Penetrometer'',
\newblock {\it Journal~of~Agricultural~Engineering~Research}, Vol.~14, No.~2, pp.~183--186, 1969.
\\

% J
\bibitem[Janik 1998]{Janik1998} % スペクトル画像を用いた土の性質の推定
\leavevmode \\L.~J.~Janik, R.~H.~Merry and J.~O.~Skjemstad:
\newblock ``Can Mid Infrared Diffuse Reflectance Analysis Replace Soil Extractions?'',
\newblock {\it Australian Journal of Experimental Agriculture}, Vol.~38, pp.~681--696, 1998.
\\ 

\bibitem[Jenkins 1985]{Jenkins1985} % コーン指数を関数で算出
\leavevmode \\S.~A.~Jenkins:
\newblock ``Clandestine Methods for the Determination of Beach Trafficability'',
\newblock Scripps Institution of Oceanography, SIO Reference Series 85-27, AD-A163-007 Reference Series, May 1985.
\\

\bibitem[Jia 2017]{Jia2017} % ハイパースペクトル画像を用いた土の種類の識別
\leavevmode \\Shengyao~Jia, Hongyang~Li, Yanjie~Wang, Renyuan~Tong and  Qing~Li:
\newblock ``Hyperspectral Imaging Analysis for the Classification of Soil Types and the Determination of Soil Total Nitrogen'',
\newblock {\it Sensors}, Vol.~17, No.~10:2252, pp.~1--14, 2017.
\\ 

% K
\bibitem[Kogure 1985]{Kogure1985} % コーン指数の誤差には2種類あり,標準正規分布に従う
\leavevmode \\K.~Kogure, Y.~Ohira and H.~Yamaguchi:
\newblock ``Basic Study of Probabilistic Approach to Prediction of Soil Trafficability -- Statistical Characteristics of Cone Index'',
\newblock {\it Journal of Terramechanics}, Vol.~22, No.~3, pp.~147--156 , 1985.
\\

\bibitem[Kruse 2000]{Kruse2000} % 分光反射率スペクトルを用いて土の支持力を分類
\leavevmode \\F.~A.~Kruse, J.~W.~Boardman and A.~B.~Lefkoff:
\newblock ``Extraction of Compositional Information for Trafficability Mapping from Hyperspectral Data'',
\newblock {\it Proceedings~of~the~SPIE 4049, Algorithms for Multispectral, Hyperspectral, and Ultraspectral Imagery VI}, pp.~261-273, 2000.
\\

\bibitem[Kruse 2003]{Kruse2003} % 地表の鉱物の分布
\leavevmode \\F.~A.~Kruse, J.~W.~Boardman and J.~F.~Huntington:
\newblock ``Comparison of Airborne Hyperspectral Data and EO-1 Hyperion for Mineral Mapping'',
\newblock {\it  IEEE Transactions on Geoscience and Remote Sensing}, Vol.~41, No.~6, pp.~1388--1400 , 2003.
\\

% L
\bibitem[Lobell 2002]{Lobell2002} % 水は近赤外の波長帯の光を吸収
\leavevmode \\D.~B.~Lobell and G.~P.~Asner:
\newblock ``Moisture Effects on Soil Reflectance'',
\newblock {\it Soil Science Society of America Journal}, Vol.~66, pp.~722--727 , 2002.
\\ 

\newpage

% M
\bibitem[Matsuo 1974]{Matsuo1974} % コーン指数の誤差には2種類あり,標準正規分布に従う
\leavevmode \\M.~Matsuo and K.~Kuroda:
\newblock ``Probabilistic Approach to Design of Embankments'',
\newblock {\it Soils and Foundations}, Vol.~14, No.~2, pp.~1--17 , 1974.
\\ 

\bibitem[Meyer 1961]{Meyer1961} % 走破性に影響を与える土の基礎的な性質
\leavevmode \\M.~P.~Meyer and S.~J.~Knight:
\newblock ``Trafficability of Soils: Soil Classification'',
\newblock U. S. Army Engineer Waterways Experiment Station, Vicksburg, Technical Memorandum, No.~3--240, 1961.
\\

\bibitem[Mulhearn 2001]{Mulhearn2001} % コーン指数を関数で算出
\leavevmode \\P.~J.~Mulhearn:
\newblock ``Methods of Obtaining Soil Strength Data for Modelling Vehicle Trafficability on Beaches'',
\newblock DSTO Aeronautical and Maritime Research Laborator, Victoria (Australia), General Document , DSTO-GD-0299, 2001.
\\

\bibitem[Mulqueen 1977]{Mulqueen1977} % コーン指数の簡単な解説
\leavevmode \\J.~Mulqueen, J.~V.~Stafford and D.~W.~Tanner:
\newblock ``Evaluation~of~Penetrometers~for~Measuring~Soil~Strength'',
\newblock {\it Journal~of~Terramechanics}, Vol.~14, No.~3, pp.~137--151, 1977.
\\

% N
\bibitem[Nolan 2003]{Nolan2003} % 走破性に影響を与える土の基礎的な性質(含水比)
\leavevmode \\M.~Nolan and D.~R.~Fatland:
\newblock ``New DEMs may Stimulate Significant Advancements in Remote Sensing of Soil Moisture'',
\newblock {\it  Eos Transactions American Geophysical Union}, Vol.~84, No.~25, pp.~233--240, 2003.
\\

% O
\bibitem[Okello 1991]{Okello1991} % 走破性に影響を与える土の基礎的な性質
\leavevmode \\J.~A.~Okello:
\newblock ``A Review of Soil Strength Measurement Techniques for
Prediction of Terrain Vehicle Performance'',
\newblock {\it Journal of Agricultural Engineering Research}, Vol.~50, pp.~129--155, 1991.
\\

% P
\bibitem[Perumpral 1987]{Perumpral1987} % コーン指数の簡単な解説とコーンペネトロメータの歴史
\leavevmode \\J.~V.~Perumpral:
\newblock ``Cone~Penetrometer~Applications~-~A~Review'',
\newblock {\it Transactions~of~the~ASAE}, Vol.~30, No.~4, pp.~939--944, 1987.
\\

\newpage

\bibitem[Prather 1970]{Prather1970} % コーンペネトロメータの派生形
\leavevmode \\O.~C.~Prather, J.~G.~Hendrick, R.~L.~Schafer:
\newblock ``An Electronic Hand-Operated Recording Penetrometer '',
\newblock {\it Transactions~of~the~ASAE}, Vol.~13, No.~1, pp.~385--386, 390, 1970.
\\

% R
\bibitem[Rankin 2010]{Rankin2010} % 画像から走破できない場所(泥)を認識
\leavevmode \\A.~L.~Rankin and L.~H.~Matthies:
\newblock ``Passive~Sensor~Evaluation~for~Unmanned~Ground~Vehicle~Mud~Detection'',
\newblock {\it Journal~of~Field~Robotics}, Vol.~27, No.~4, pp.~473--490, 2010.
\\

\bibitem[Rossel 2006]{Rossel2006} % スペクトル画像を用いた土の性質の推定
\leavevmode \\R.~A.~V.~Rossel, D.~J.~J.~Walvoort, A.~B.~McBratney, L.~J.~Janik and J.~O.~Skjemstad:
\newblock ``Visible, Near Infrared, Mid Infrared or Combined Diffuse Reflectance Spectroscopy for Simultaneous Assessment of Various Soil Properties'',
\newblock {\it Geoderma}, Vol.~131, No.~1--2, pp.~59--75, 2006.
\\

% S
\bibitem[Shaw 2002]{Shaw2002} % スペクトル画像の説明
\leavevmode \\G.~A.~Shaw and D.~Manolakis:
\newblock ``Signal Processing for Hyperspectral Image Exploitation'',
\newblock {\it IEEE Signal Processing Magazine}, Vol.~19, No.~1, pp.~12--16, 2002.
\\

\bibitem[Shaw 2003]{Shaw2003} % スペクトル画像の説明
\leavevmode \\G.~A.~Shaw and Hsiao-hua~K.~Burke:
\newblock ``Spectral Imaging for Remote Sensing'',
\newblock {\it Lincoln Laboratory Journal}, Vol.~14, No.~1, pp.~3--28, 2003.
\\

\bibitem[Shoop 1993]{Shoop1993} % 走破性に影響を与える土の基礎的な性質
\leavevmode \\S.~A.~Shoop:
\newblock ``Terrain Characterization for Trafficability'',
\newblock U.S. Army Cold Regions Research and Engineering Laboratory, Hanover, CRREI. Report 93-6, 1993.
\\

% \bibitem[Sopher 2016]{Sopher2016} % 名前が省略されている論文と合わせる必要がある
% \leavevmode \\Ariana~M.~Sopher, Sally~A.~Shoop, Jesse~M.~Stanley,~Jr., and Brian~T.~Tracy:
% \newblock ``Image Analysis and Classification Based on Soil Strength'',
% \newblock U.S.~Army~Engineer~Research~and~Development~Center,~Cold~Regions~Research~and~Engineering~Laboratory, Hanover, Technical Report, ERDC/CRREL TR-16-13, August 2016.
% \\
\bibitem[Sopher 2016]{Sopher2016} % 可視光の分光反射率スペクトルを用いて土の支持力を分類
\leavevmode \\A.~M.~Sopher, S.~A.~Shoop, J.~M.~Stanley,~Jr. and B.~T.~Tracy:
\newblock ``Image Analysis and Classification Based on Soil Strength'',
\newblock U.S.~Army~Engineer~Research~and~Development~Center,~Cold~Regions~Research~and~Engineering~Laboratory, Hanover, Technical Report, ERDC/CRREL TR-16-13, 2016.
\\

\newpage

% T
\bibitem[Tian 2015]{Tian2015} % 水は近赤外の波長帯の光を吸収
\leavevmode \\J.~Tian and W.~D.~Philpot:
\newblock ``Relationship between Surface Soil Water Content, Evaporation Rate, and Water Absorption Band Depths in SWIR Reflectance Spectra'',
\newblock {\it Remote Sensing of Environment}, Vol.~169, pp.~280--289, 2015.
\\

\bibitem[Tominaga 1999]{Tominaga1999} % マルチスペクトル画像を使用した分類 
\leavevmode \\Shoji~Tominaga:
\newblock ``Spectral~Imaging~by~a~Multichannel~Camera'',
\newblock {\it Journal~of~Electronic~Imaging}, Vol.~8, No.~4, pp.~332--341, 1999.
\\

% U
\bibitem[Underwood 2003]{Underwood2003} % 生態系の監視
\leavevmode \\E.~Underwood, S.~Ustin and D.~DiPietro:
\newblock ``Mapping Nonnative Plants Using Hyperspectral Imagery'',
\newblock {\it Remote Sensing of Environment}, Vol.~86, No.~2, pp.~150--161, 2003.
\\

\bibitem[Ustin 2004]{Ustin2004} % 生態系の調査
\leavevmode \\S.~L.~Ustin, D.~A.~Roberts, J.~A.~Gamon, G.~P.~Asner and R.~O.~Green:
\newblock ``Using Imaging Spectroscopy to Study Ecosystem Processes and Properties'',
\newblock {\it BioScience}, Vol.~54, No.~6, pp.~523--534, 2004.
\\

% W
\bibitem[Waterways Experiment Station 1948]{WES1948} % コーンペネトロメータの詳しい解説
\leavevmode \\Waterways Experiment Station:
\newblock ``Trafficability of Soils: Development of Testing Instruments'',
\newblock U.S. Waterways Experiment Station, Vicksburg, Waterways Experiment Station (U.S.). Technical Memorandum, No.~3--240, 
Third Supplement, 1948.
\\

% Z
\bibitem[Zacny 2010]{Zacny2010} % 宇宙用のローバにペネトロメータをつけてコーン指数測定
\leavevmode \\K.~Zacny, J.~Wilson, J.~Craft, V.~Asnani, H.~Oravec, C.~Creager, J.~Johnson and T.~Fong:
\newblock ``Robotic~Lunar~Geotechnical~Tool'',
\newblock {\it Proceeding~of~the~2010~ASCE~Earth~and~Space, Honolulu, HI}, March 2010.
\\

% 後で更なる具体例を解説
 

% \bibitem[McCarty 2002]{McCarty2002} % スペクトル画像を用いた土の性質の推定
% \leavevmode \\G.~W.~McCarty, J.~B.~Reeves~III, V.~B.~Reeves, R.~F.~Follett and J.~M.~Kimble:
% \newblock ``Mid-Infrared and Near-Infrared Diffuse Reflectance Spectroscopy for Soil Carbon Measurement'',
% \newblock {\it Soil Science Society of America Journal}, Vol.~66, No.~2, pp.~640--646, 2002.
% \\ 



% \bibitem[Cozzolino 2003]{Cozzolino2003} % スペクトル画像を用いた土の性質の推定
% \leavevmode \\D.~Cozzolino and A.~Moro\'n:
% \newblock ``The Potential of Near-infrared Reflectance Spectroscopy to Analyse Soil Chemical and Physical Characteristics'',
% \newblock {\it Journal of Agricultural Science}, Vol.~140, No.~1, pp.~65--71, 2003.
% \\ 

% \newpage



% \bibitem[Tominaga 1999]{Tominaga1999} % スペクトル画像の解説 被った
% \leavevmode \\S.~Tominaga:
% \newblock ``Spectral Imaging by a Multichannel Camera'',
% \newblock {\it Journal of Electronic Imaging}, Vol.~8, No.~4, pp.~332--341, 1999.
% \\ 

%Bundle Adjustment -- A Modern Synthesis
%Bill Triggs, Philip F. McLauchlan, Richard I. Hartley, and Andrew W. Fitzgibbon:
%Vision Algorithms: Theory and Practice, pp.~298--372, 2000.

%absolute pose
%Absolute Pose for Cameras Under Flat Refractive Interfaces


\end{mythebibliography}