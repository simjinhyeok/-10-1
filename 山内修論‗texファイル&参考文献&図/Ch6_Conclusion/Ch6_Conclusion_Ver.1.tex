\chapter{結論}
\thispagestyle{empty}
\label{ch:Conclusion}
\minitoc

\newpage
%%%%%%%%%%%%%%%%%%%%%%%%%%%%%%%%%%%%%%%%%%%%%%%%%%%%%%%%%%%%%%%%%%%%%%%%%%%%%%%
%==============================================================================
%結論
%==============================================================================
\section{結論}

本研究では,ハイパースペクトル画像とマルチスペクトル画像を用いて,土の種類の識別と含水比の推定を行い,
識別した土の種類と推定した含水比からコーン指数を推定することによって,非接触で走破性を判定する手法を提案した.

第\ref{ch:Introduction}章では,
本研究の背景,先行研究,および目的について述べた.
%
まず最初に,本研究の背景として,土砂災害の発生現場における
建設機械の走破性判定の必要性を述べた.
%
次に,走破性判定のための先行研究に触れ,
災害現場での走破性判定には,
非接触で走破性を判定する手法が求められているが,
従来提案されてきた,
非接触での走破性判定のために,
走破性の指標の1つであるコーン指数を非接触で推定する手法では,
コーン指数に大きな影響を与える土の種類と含水比の双方に注目した手法は
確認されていないことを述べた.
%
そこで,本研究の目的を,土の種類と含水比の双方に注目した,
建設機械のための非接触での走破性判定とすることを述べた.


第\ref{ch:PrinciplesOfMethod}章では,
本研究で提案した,
スペクトル画像を用いることで
コーン指数を非接触で推定することにより,
非接触での走破性判定を行う手法の原理の詳細について述べた.
%
まず最初に,土の性質を示す土質パラメータについて解説し,
コーン指数がそのうちの1つである
ことを述べた.
また,コーン指数が他の土質パラメータである
土の粒子の鉱物組成,有機物含有量,粒度分布,球形率,含水比
から大きな影響を受けることを述べた.
% 
次に,これらの土質パラメータのうち,
外部の状況に左右されない土に固有の土質パラメータである
土の粒子の鉱物組成,有機物含有量,粒度分布,球形率,含水比が同じ土を
同じ種類であると定義すると,
土の種類と含水比から
コーン指数を推定できることを述べた.
%
また,分光反射率スペクトルが物質の種類と状態によって変化することを述べ,
その性質を利用することによって,分光反射率スペクトルから
土の種類と状態を非接触に推定することができることを述べた.
%
次に,建設機械の走破性判定のために
広い範囲のコーン指数を推定する必要があるため,
本研究では,広い範囲の分光反射率スペクトルを
取得できるスペクトル画像を用いることを述べた.
%
最後に,画像を用いて非接触での走破性の判定を行うにあたり,
土の表面の情報しか見ないことについての議論を行い,
画像を用いて土の表面しか見なくとも走破性の判定は
可能であることを示した.

第\ref{ch:SoilTypeDiscrimination}章では,
提案手法において最初のステップである,
土の種類をスペクトル画像から推定するステップの詳細について述べた.
%
まず最初に,多くの土の種類を分光反射率スペクトルを用いて推定するためには,
波長幅の短い波長帯を多数取得できる分光反射率スペクトルが必要となることを述べ,
そのために,波長分解能が高く入射光を非常に多くの波長帯に分光させるハイパースペクトル画像から
分光反射率スペクトルを取得することを述べた.
%
次に,その分光反射率スペクトルから土の種類を識別するために使用した
3層のニューラルネットワークの詳細について解説した.
% 
最後に,ハイパースペクトル画像から取得した分光反射率スペクトルを用いて
土の種類を識別することが
できるか確認するための検証実験について解説し,
その結果から,ハイパースペクトル画像を用いた土の種類の識別は可能であることを確認した.

第\ref{ch:WaterContentEstimation}章では,
提案手法において土の種類の識別の次に来るステップである,
含水比をスペクトル画像から推定するステップの詳細について述べた.
%
まず最初に,本研究では,
水が近赤外の光を吸収するという性質を利用して,
水が光を吸収する近赤外の波長帯の分光反射率と,水が光を吸収しない
波長帯の分光反射率を分光反射率スペクトルから取得し,
その2つの分光反射率の差から含水比を推定することを述べた.
%
また,水は近赤外の広い範囲の光を吸収するため,入射光を分光させる際に
近赤外の広い範囲の光を1つの波長帯として分光させる必要があることに言及し,
波長分解能が低く,近赤外の広い範囲の光を1つの波長帯として分光して記録できる
マルチスペクトル画像を使用を用いて上記の2つの分光反射率を取得することを述べた.
% マルチスペクトル画像のより詳細な説明? 前の原理の章でやった.
次に,上記で述べた2つの分光反射率の差から含水比を
推定するために,指数近似曲線をフィッティングさせることを述べた.
%
最後に,マルチスペクトル画像から含水比を推定できるか確認するするための
検証実験について解説し,その結果から,
マルチスペクトル画像を用いた含水比の推定は可能であることを確認した.
% 実験結果の詳細? 図がメインの結果となり,決定係数だけを載せても読者には理解してもらえないかも?

第\ref{ch:ConeIndexEstimation}章では,
提案手法の最後のステップである,
土の種類と含水比から,コーン指数を推定するステップの詳細について述べた.
% 
まず最初に,本研究では,
コーン指数と含水比の関係を用いることによって,
第\ref{ch:SoilTypeDiscrimination}章で識別した土の種類と
第\ref{ch:WaterContentEstimation}章で推定した含水比からコーン指数を推定することを述べた.
また,予め土の種類ごとに含水比を変えながらコーン指数を測定しておくことによって,
土の種類によって異なる含水比とコーン指数の関係を記録することにも言及した.
% 
次に,ハイパースペクトル画像から識別された土の種類のコーン指数と含水比の関係を参照して,
推定された含水比からコーン指数を推定する手法の詳細について解説した.
% 
最後に,屋外で撮影したハイパースペクトル画像とマルチスペクトル画像から
コーン指数の推定ができるか確認するための検証実験を行った.
% ハイパースペクトル画像とマルチスペクトル画像の輝度値と変動係数の条件?
その検証実験の結果から,コーン指数を推定する土の,含水比に対するコーン指数の変動が穏やかな含水比の範囲においては,
ハイパースペクトル画像とマルチスペクトル画像を用いることによる非接触でのコーン指数の推定が
可能であることを確認した.
また,建設機械が走破できる限界のコーン指数を閾値とすることによって,
非接触に推定したコーン指数に基づく走破性判定が可能であることも確認した.

本研究により,ハイパースペクトル画像とマルチスペクトル画像を用いることによって,
走破性に大きく影響する土の種類と含水比の双方を考慮した,
建設機械のための非接触での走破性判定を実現した.

\newpage


%==============================================================================
%今後の展望
%==============================================================================
\section{今後の展望}

本研究では,スペクトル画像を用いることによって
土の種類の識別と含水比の推定を行い,その2つからコーン指数を推定する手法を提案した.
これにより,土の種類と含水比の双方を考慮した非接触での走破性判定が可能となった.
一方で,コーン指数推定の対象となる土の,含水比に対するコーン指数の変動が激しい含水比の範囲においては,
含水比の推定にわずかな誤差が生じても
コーン指数の実測値と推定値の差が大きくなるため,
コーン指数の推定値の信頼性が低下する.
信頼性の低いコーン指数の推定値に基づいて走破性を判定すると,その走破性判定の信頼性も低くなるため,
コーン指数推定の対象となる土の,含水比に対するコーン指数の変動が激しい含水比の範囲においては,非接触での走破性判定が困難になる.
そこで,今後の課題としては,
含水比に対するコーン指数の変動が激しい含水比の範囲においても
コーン指数の推定を行うことが挙げられる.
そのためには,含水比の推定精度をさらに向上させる手法を提案する必要がある.
% また,多くの粘性土において,
% ある一定の含水比に達するとコーン指数が急落する現象がみられるため,
% この現象の原因を特定し,それに関連する物理量を非接触に推定することも必要であると考えられる.

また,本研究ではスペクトル画像から取得した分光反射率スペクトルを用いてコーン指数を推定するため,
スペクトル画像から土と校正シートのスペクトルを確実に取得する必要がある.
そのため,
% スペクトル画像が明るすぎたり暗すぎたり,あるいは
土や校正シートからの反射光に別の物質の反射光が混合して
分光反射率の計算ができなくなると,コーン指数の推定も困難になる.
本研究では,分光反射率スペクトルを確実に取得するため,
暫定的にスペクトル画像の輝度値と変動係数に閾値を設けて,不適切なスペクトル画像を除外したが,
今後は,土と校正シートの分光反射率スペクトルを取得できる環境の条件の限界を
より明確にすることによって,推定したコーン指数の信頼性を担保する必要がある.

% 次に,スペクトル画像は入射光を4以上の波長帯に分光させて光の強さを記録するため,
% 一般的なRGB画像に比べて1波長帯あたりの光の強さは少なくなってしまう.
% 従って,どこまで撮影する環境が暗いと撮影出来なくなってしまうのか,
% 限界を確認することも今後の課題となる.

最後に,本研究の提案手法は,それぞれの土の種類のハイパースペクトル画像とマルチスペクトル画像,ならびにコーン指数と含水比の関係を
事前に取得できていることが前提となる.
従って,ハイパースペクトル画像,マルチスペクトル画像,ならびにコーン指数と含水比の関係を取得できている土の種類が少ないと
% 本研究
本研究の提案手法の有効性が低下する.
そこで,本研究での提案手法の有効性を向上させるために,今後はより多くの土の種類のスペクトル画像ならびに含水比とコーン指数の関係の測定および記録
を行う必要がある.

% 今後の課題としては,含水比の推定精度を向上させることにより,含水比に 
% 対するコーン指数の変化が大きい場合においても,コーン指数の推定精度を向上 
% させることが挙げられる.
\newpage

%%%%%%%%%%%%%%%%%%%%%%%%%%%%%%%%%%%%%%%%%%%%%%%%%%%%%%%%%%%%%%%%%%%%%%%%%%%%%%%%
%%%% Local Variables:
%%%% mode: katex
%%%% TeX-master: "../thesis"
%%%% End:
